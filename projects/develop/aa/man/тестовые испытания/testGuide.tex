\documentclass[10pt,a4paper]{report}
\usepackage{cmap}
\usepackage[russian]{babel}
\usepackage{amsmath}
\usepackage{amsfonts}
\usepackage{amssymb}
\usepackage{graphicx}
\usepackage{listings}
\usepackage{hyperref}
\usepackage{float}
\usepackage{fontspec}
\setmainfont{Times New Roman}
\setmonofont{Arial}

\lstset{
	inputencoding=utf8x,
	extendedchars=\true,
	frame=single,
	breaklines=true,
	numbers=left,
	keepspaces = true}

\voffset -24.5mm
\hoffset -5mm
\textwidth 173mm
\textheight 240mm
\oddsidemargin=0mm \evensidemargin=0mm

\author{Абдуллин Азат}
\title{Программа и методика испытаний}
\begin{document}
	\maketitle
	\renewcommand{\thesection}{\arabic{section}}
	\tableofcontents
	\pagebreak
	
\section{Объект испытаний}
Объектом испытаний является программное изделие "Мобильный оконный менеджер XXwm", которое является оконным менеджером Wayland для ОС ArchLinux. Данное программное обеспечение позволяет запускать пользовательские графические приложения.
		
\section{Цель испытаний}
Целью испытаний является проверка корректности функционирования программного обеспечения, и проверка реализации заявленных в техническом задании функциональных требований, предъявляемых к данному программному обеспечению.
	
\section{Требования к программе}
Программное обеспечение "Мобильный оконный менеджер XXwm" должно:
\begin{itemize}
\item Запускать системные приложения строки состояния и рабочего стола при старте
\item Предоставлять возможность запуска пользовательских графических приложений
\item Предоставлять возможности перемещения и изменения размеров пользовательских окон
\item Обрабатывать нажатия комбинаций клавиш
\end{itemize}
		
Методы проверки приведенных выше требований описаны в данном документе.
		
\section{Требования к программной документации}
Для проведения программы испытаний необходимы следующие документы на программное обеспечение "Мобильный оконный менеджер XXwm":
\begin{itemize}
\item Техническое задание;
\item Текст программы;
\item Руководство системного программиста.
\end{itemize}
	
В техническом задании описываются функциональные требования,  которые реализовывает исследуемое программное обеспечение и требуемые параметры операционной среды.
		
В тексте программы приводится символическая запись программного обеспечения на исходном языке, которая поясняет реализацию функциональных требований.
		
В руководстве системного программиста описаны способы инсталляции и деинсталляции исследуемого программного обеспечения.
		
\section{Методы испытаний}
\begin{table}[H]
\caption{Сообщения системному администратору}
\label{tabular:timesandtenses}
\begin{center}
\begin{tabular}{| p{0.1\linewidth} | p{0.4\linewidth} | p{0.4\linewidth} |}
\hline
\textbf{Номер метода} & \textbf{Порядок выполнения} & \textbf{Положительный результат проверки} \\
\hline
1 & Компиляция исходных кодов. Для этого необходимо скопировать  исполняемый файл с компилятором yasm в директорию с исходными кодами и выполнить команды: 
\begin{verbatim}
cmake .
make
\end{verbatim} &  Сборка исполняемых файлов должна завершиться без ошибок. В рабочем каталоге должен появиться исполняемый файл xxwm.\\
\hline
2 & Установить оконный менеджер, согласно инструкции, и перезагрузить машину &  В экранном менеджере при выборе оконного менеджера должен появиться пункт XXwm\\
\hline
3 & Запустить оконный менеджер & Должен запуститься оконный менеджер. В нем должны запуститься системные приложения рабочего стола и строки состояния\\
\hline
4 & Запустить оконный менеджер и нажать комбинацию клавиш CTRL+Esc & Оконный менеджер завершит свою работу\\
\hline
5 & Запустить оконный менеджер и нажать комбинацию клавиш CTRL+q & Ничего не должно произойти \\
\hline
6 & Нажать комбинацию клавиш CTRL+Enter & Запустится терминал\\
\hline
7 & Нажать комбинацию клавиш CTRL+стрелка вниз & На экране должен отобразиться рабочий стол\\
\hline
8 & Нажать на иконку на рабочем столе & Должно запуститься соответствующее приложение\\
\hline
9 & Нажать комбинацию клавиш CTRL+q & Запущенное клиентское приложение должно закрыться\\
\hline
10 & Зажать клавишу CTRL, нажать на ЛКМ и передвигать указатель мыши & Активное окно должно перемещаться по экрану в соответствии с передвижением указателя мыши\\
\hline
11 & Зажать клавишу CTRL, нажать на ПКМ и передвигать указатель мыши & Активное окно должно изменять свой размер в соответствии с передвижением указателя мыши\\
\hline
\end{tabular}
\end{center}
\end{table}
		
\end{document}