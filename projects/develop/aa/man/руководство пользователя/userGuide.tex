\documentclass[10pt,a4paper]{report}
\usepackage{cmap}
\usepackage[russian]{babel}
\usepackage{amsmath}
\usepackage{amsfonts}
\usepackage{amssymb}
\usepackage{graphicx}
\usepackage{listings}
\usepackage{hyperref}
\usepackage{fontspec}
\setmainfont{Times New Roman}
\setmonofont{Arial}

\lstset{
	inputencoding=utf8x,
	extendedchars=\true,
	frame=single,
	breaklines=true,
	numbers=left,
	keepspaces = true}

\voffset -24.5mm
\hoffset -5mm
\textwidth 173mm
\textheight 240mm
\oddsidemargin=0mm \evensidemargin=0mm

\author{Абдуллин Азат}
\title{Руководство пользователя}
\begin{document}
\maketitle
\renewcommand{\thesection}{\arabic{section}}
\section{Введение}
Данная программа предназначена для запуска графических пользовательских приложений 	в ОС ArchLinux на платформе Raspberry Pi Zero. Для запуска программы в системе должны быть установлены пакеты Wayland и WLC.
	
\section{Подготовка к работе}
Первым шагом при подготовке оконного менеджера к работе является компиляция исходных кодов программы в бинарные файлы. Для компиляции необходимо воспользоваться системой сборки CMake, версии, не ниже 3.7. Так же в системе должны быть установлены компилятор GCC и система сборки Make.
Компиляция осуществляется следующими командами:
\begin{verbatim}
cmake .
make
\end{verbatim}
В результате, в рабочей директории появится исполняемый файл оконного менеджера xxwm.
					
Для установки программы необходимо иметь права администратора. Для установки необходимо:
\begin{itemize}
\item Поместить исполняемый файл оконного менеджера в каталог исполняемых файлов:
\begin{verbatim}
cp xxwm /usr/bin/
\end{verbatim}
\item Поместить стандартный конфигурационный файл в директорию хранения конфигураций:
\begin{verbatim}
cp config.ini ~/.config/xxwm
\end{verbatim}
\item Поместить файл ярлыка для экранного менеджера в директорию ярлыков оконных менеджеров:
\begin{verbatim}
cp xxon.desktop /usr/share/wayland-sessions/
\end{verbatim}
\end{itemize}
Для настройки программы необходимо в конфигурационном файле ~/.config/xxwm указать пути к исполняемым файлам строки состояния и рабочего стола

Для запуска оконного менеджера необходимо перезагрузить машину и в меню экранного менеджера выбрать пункт XXon.
		
\section{Описание операций}
При запуске оконного менеджера будут также запущены системные приложения строки состояния. В оконном менеджере пользователю доступны следующие комбинации клавиш:
\begin{itemize}
\item CTRL+q --- закрыть активное окно (если это не системное приложение)
\item CTRL+стрелка вниз --- переключиться на следующее окно
\item CTRL+Enter --- запустить терминал
\item CTRL+Esc --- завершить работу оконного менеджера
\end{itemize}

При нажатии на иконку на рабочем столе запускается соответствующее приложение. Пользователю доступны возможности перемещения окна по экрану (зажать CTRL+ЛКМ и перемещать указатель мыши) и изменения размеров окна (зажать CTRL+ПКМ и перемещать указатель мыши).

\section{Аварийные ситуации}
При неправильной настройке конфигурационного файла при запуске оконного менеджера может возникнуть ошибка и системные приложения не запустятся. Для решения данной проблемы необходимо указать корректные пути к исполняемым файлам системных приложений в конфигурационном файле.
\end{document}